% Options for packages loaded elsewhere
\PassOptionsToPackage{unicode}{hyperref}
\PassOptionsToPackage{hyphens}{url}
%
\documentclass[
  12pt,
]{article}
\usepackage{amsmath,amssymb}
\usepackage{iftex}
\ifPDFTeX
  \usepackage[T1]{fontenc}
  \usepackage[utf8]{inputenc}
  \usepackage{textcomp} % provide euro and other symbols
\else % if luatex or xetex
  \usepackage{unicode-math} % this also loads fontspec
  \defaultfontfeatures{Scale=MatchLowercase}
  \defaultfontfeatures[\rmfamily]{Ligatures=TeX,Scale=1}
\fi
\usepackage{lmodern}
\ifPDFTeX\else
  % xetex/luatex font selection
\fi
% Use upquote if available, for straight quotes in verbatim environments
\IfFileExists{upquote.sty}{\usepackage{upquote}}{}
\IfFileExists{microtype.sty}{% use microtype if available
  \usepackage[]{microtype}
  \UseMicrotypeSet[protrusion]{basicmath} % disable protrusion for tt fonts
}{}
\makeatletter
\@ifundefined{KOMAClassName}{% if non-KOMA class
  \IfFileExists{parskip.sty}{%
    \usepackage{parskip}
  }{% else
    \setlength{\parindent}{0pt}
    \setlength{\parskip}{6pt plus 2pt minus 1pt}}
}{% if KOMA class
  \KOMAoptions{parskip=half}}
\makeatother
\usepackage{xcolor}
\usepackage[margin=1in]{geometry}
\usepackage{longtable,booktabs,array}
\usepackage{calc} % for calculating minipage widths
% Correct order of tables after \paragraph or \subparagraph
\usepackage{etoolbox}
\makeatletter
\patchcmd\longtable{\par}{\if@noskipsec\mbox{}\fi\par}{}{}
\makeatother
% Allow footnotes in longtable head/foot
\IfFileExists{footnotehyper.sty}{\usepackage{footnotehyper}}{\usepackage{footnote}}
\makesavenoteenv{longtable}
\usepackage{graphicx}
\makeatletter
\def\maxwidth{\ifdim\Gin@nat@width>\linewidth\linewidth\else\Gin@nat@width\fi}
\def\maxheight{\ifdim\Gin@nat@height>\textheight\textheight\else\Gin@nat@height\fi}
\makeatother
% Scale images if necessary, so that they will not overflow the page
% margins by default, and it is still possible to overwrite the defaults
% using explicit options in \includegraphics[width, height, ...]{}
\setkeys{Gin}{width=\maxwidth,height=\maxheight,keepaspectratio}
% Set default figure placement to htbp
\makeatletter
\def\fps@figure{htbp}
\makeatother
\setlength{\emergencystretch}{3em} % prevent overfull lines
\providecommand{\tightlist}{%
  \setlength{\itemsep}{0pt}\setlength{\parskip}{0pt}}
\setcounter{secnumdepth}{-\maxdimen} % remove section numbering
\usepackage{setspace}\doublespacing
\ifLuaTeX
  \usepackage{selnolig}  % disable illegal ligatures
\fi
\IfFileExists{bookmark.sty}{\usepackage{bookmark}}{\usepackage{hyperref}}
\IfFileExists{xurl.sty}{\usepackage{xurl}}{} % add URL line breaks if available
\urlstyle{same}
\hypersetup{
  pdftitle={Data Exploration},
  pdfauthor={Miao Fu},
  hidelinks,
  pdfcreator={LaTeX via pandoc}}

\title{Data Exploration}
\author{Miao Fu}
\date{2023-12-03}

\begin{document}
\maketitle

\hypertarget{descriptive-summary-statistics-for-all-variables}{%
\section{Descriptive summary statistics for all
variables}\label{descriptive-summary-statistics-for-all-variables}}

Two table with summary information on the descriptive statistics of all
variables are listed below. The frequency and percentage of each
categories in each categorical variable is listed out. For each numeric
variable, the table includes values of mean, median, standard deviation,
minimum, maximum, Q1 and Q3 values.

\hypertarget{categorical-variables}{%
\subsection{Categorical Variables}\label{categorical-variables}}

\begin{longtable}[]{@{}llrr@{}}
\toprule\noalign{}
variable & category & count & percent \\
\midrule\noalign{}
\endhead
\bottomrule\noalign{}
\endlastfoot
gender & female & 315 & 53.662692 \\
gender & male & 272 & 46.337308 \\
ethnic\_group & group A & 50 & 8.517888 \\
ethnic\_group & group B & 123 & 20.954003 \\
ethnic\_group & group C & 174 & 29.642249 \\
ethnic\_group & group D & 155 & 26.405451 \\
ethnic\_group & group E & 85 & 14.480409 \\
parent\_educ & associate's degree & 128 & 21.805792 \\
parent\_educ & bachelor's degree & 71 & 12.095400 \\
parent\_educ & high school & 122 & 20.783646 \\
parent\_educ & master's degree & 39 & 6.643952 \\
parent\_educ & some college & 116 & 19.761499 \\
parent\_educ & some high school & 111 & 18.909710 \\
lunch\_type & free/reduced & 206 & 35.093697 \\
lunch\_type & standard & 381 & 64.906303 \\
test\_prep & completed & 208 & 35.434412 \\
test\_prep & none & 379 & 64.565588 \\
parent\_marital\_status & divorced & 92 & 15.672913 \\
parent\_marital\_status & married & 343 & 58.432709 \\
parent\_marital\_status & single & 137 & 23.339012 \\
parent\_marital\_status & widowed & 15 & 2.555366 \\
practice\_sport & never & 68 & 11.584327 \\
practice\_sport & regularly & 218 & 37.137990 \\
practice\_sport & sometimes & 301 & 51.277683 \\
is\_first\_child & no & 192 & 32.708688 \\
is\_first\_child & yes & 395 & 67.291312 \\
transport\_means & private & 229 & 39.011925 \\
transport\_means & school\_bus & 358 & 60.988075 \\
wkly\_study\_hours & \textless{} 5 & 154 & 26.235094 \\
wkly\_study\_hours & \textgreater{} 10 & 104 & 17.717206 \\
wkly\_study\_hours & 5-10 & 329 & 56.047700 \\
\end{longtable}

\hypertarget{numeric-variables}{%
\subsection{Numeric Variables}\label{numeric-variables}}

\begin{longtable}[]{@{}lrrrrrrr@{}}
\toprule\noalign{}
variable & mean & median & sd & minimum & maximum & q1 & q3 \\
\midrule\noalign{}
\endhead
\bottomrule\noalign{}
\endlastfoot
nr\_siblings & 2.139693 & 2 & 1.481712 & 0 & 7 & 1 & 3 \\
math\_score & 66.676320 & 67 & 16.113744 & 0 & 100 & 56 & 78 \\
reading\_score & 69.846678 & 70 & 15.166662 & 17 & 100 & 60 & 81 \\
writing\_score & 68.901192 & 69 & 15.550000 & 10 & 100 & 58 & 79 \\
\end{longtable}

\hypertarget{distribution-of-the-outcomes}{%
\section{Distribution of the
outcomes}\label{distribution-of-the-outcomes}}

The outcome of this study includes the following variables: maths
scores, reading scores, and writing scores. QQplots of the outcome
variables are created to explore the distribution of each score. QQplot
compares the quantiles of the data against the quantiles of a normal
distribution. According the plots, majority of the data points of all
three scores follow the straight qqline, which indicates they follow the
normal distribution. However, there are some deviations from the line on
the two ends of the distribution, which indicates the distributions
might have heavier tails than normal distribution. Or, there might be
skewness or outliers in the dataset. To further explore the distribution
of outcomes, histograms and boxplots for the scores were incorporated.
As suggested by the histograms and boxplots, all three scores are
left-skewed with outliers on the left side of the distribution.

\includegraphics{data-exploration_files/figure-latex/unnamed-chunk-5-1.pdf}
\includegraphics{data-exploration_files/figure-latex/unnamed-chunk-5-2.pdf}

\hypertarget{potential-transformations}{%
\section{Potential transformations}\label{potential-transformations}}

Potential transformations that may help further prepare the variables
for later analysis were tested. With the expectation to normalize
distribution and minimize skewness and impact of outliers, three types
of transformations were tested: 1) Natural logarithm 2) Square Root 3)
Inverse. The resulting plots are plotted in histograms shown below.
There is no apparent improvement on the distribution of the outcome
through the three transformations mentioned. Thus, orignial outcome data
were chosen to be used in following statistical modeling steps.

\includegraphics{data-exploration_files/figure-latex/unnamed-chunk-6-1.pdf}
\includegraphics{data-exploration_files/figure-latex/unnamed-chunk-6-2.pdf}
\includegraphics{data-exploration_files/figure-latex/unnamed-chunk-6-3.pdf}

\hypertarget{pairwise-relationships}{%
\section{Pairwise relationships}\label{pairwise-relationships}}

\includegraphics{data-exploration_files/figure-latex/unnamed-chunk-7-1.pdf}

By plotting our the pairwise correlation between variables, there is
apparent linearity among the three scores. Other correlation
coefficients are relatively small, indicating weak linear relationship
between the variables.

\end{document}
